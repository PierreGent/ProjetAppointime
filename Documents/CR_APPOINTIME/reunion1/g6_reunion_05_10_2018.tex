\documentclass[a4paper]{article}
\usepackage[latin1]{inputenc}
\usepackage[T1]{fontenc}
\usepackage[french]{babel}
\usepackage[pdftex]{graphicx}
\usepackage{amsfonts}
\usepackage{tabularx}
\usepackage{float}

\title{Compte-rendu de r�union du 05/10/2018}
\author{Gentile Pierre}
\date{\today}

\begin{document}

\maketitle


\begin{itemize}
\item Lieux : Campus Carnot
\item Heure de d�but : 13h00
\item Heure de fin : 13h45
\item Parcicipant : Gentile Pierre, Didier-Roche Fran�ois
\end{itemize}

\newpage

\section{Points abord�s lors de la r�union}

\subsection{Contraintes de chaque membre du groupe}
\subsubsection{Cr�naux de travail}
Didier-Roche Fran�ois : N'a pas de contraintes sur les cr�naux de
travail mais pr�f�re travailler le soir.

Gentile Pierre : Ne peux pas travailler le vendredi, samedi et dimanche soir de 18h � 23h car job �tudiant. Pr�f�re travailler le soir.

\subsubsection{Points forts et points faibles de chaques membres du groupe}
Didier-Roche Fran�ois : 
Points forts : Algorithmique, d�boguage.
Points faibles : Design, propret�e du code source.

Gentile Pierre :
Points forts : Gestion de base de donn�e, design.
Points faibles : D�boguage, propret�e du code source.

\subsection{Choix du nom du projet et de l'entreprise}

Comme nom de projet, nous avons choisis Apointime en ref�rence � la fonction de notre application : Prendre un rendez-vous en gagnant du temps.
Comme nom d'entreprise, nous avons choisis Pingouin.

\subsection{Reflexion sur le support de l'application}
Nous avons r�fl�chi sur la n�cessit� de faire une application mobile
et un site web plut�t qu'un site web responsive. En effet, une
application mobile serait n�cessaire uniquement si nous trouvons des
arguments pour produire des interactions utiles entre le mobile et
l'application : notifications, alarmes, map ... Nous avons mis cette
question en attente jusqu'� la definition des diff�rentes
fonctionnalit�s.

 
 
\end{document}