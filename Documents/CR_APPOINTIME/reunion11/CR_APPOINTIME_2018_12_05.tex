\documentclass[a4paper]{article}
\usepackage[utf8]{inputenc}
\usepackage[T1]{fontenc}
\usepackage[french]{babel}
\usepackage[pdftex]{graphicx}
\usepackage{amsfonts}
\usepackage{tabularx}
\usepackage{float}
\usepackage{fancyhdr}

\fancyhead[L]{Entreprise : Pingouin}
\fancyhead[R]{Projet : Appointime}

\title{Compte-rendu de réunion du 13/12/2018}
\author{Didier Roche François}
\date{\today}

\begin{document}
\renewcommand{\contentsname}{Ordre du jour}
\maketitle
\thispagestyle{fancy}


\begin{itemize}
\item Lieu : Saint étienne
\item Heure de début :17h
\item Heure de fin : 17h25
\item Parcicipants : Gentile Pierre, Didier-Roche François
\end{itemize}

\bigbreak
\bigbreak
\bigbreak

\tableofcontents


\newpage

\section{Retour sur les travaux menés durant la semaine}
\subsection{Redaction du document de conception générale}
Le document de conception générale a été commencé. D'une part, nous avons réfléchi à la forme de ce dernier et produit son squelette. D'autre part, nous avons créé plusieurs diagrammes de cas d'utilisation, de séquence et d'activitée.


\section{Les directives jusqu'au rendu des documents}
\subsection{Rédaction du document de conception génerale}
Étant donné que nous avons la majorité des diagrammes ainsi qu'un squelette de notre document de conception générale, nous pensons pouvoir le terminer d'ici samedi. En effet, il ne nous reste plus qu'à ajouter les diagrammes ainsi qu'à en modifier certains qui ont besoin de certaines évolutions.
\subsection{Rédaction d'un premier manuel utilisateur}
Nous allons devoir trouver des exemples de manuel d'utilisateur et créer des shémas d'interface permettant de comprendre comment utiliser notre système. Nous pensons que ce document sera rapide à effectuer car notre application comporte peut de pages.

\subsection{Rédaction du cahier de test d'integration}
Nous n'avons aucune idée du contenu de ce document. Nous devrons donc effectuer des recherche et nous espérons que ce document ne sera pas trop long à rédiger au vu du temps qu'il nous reste.

\section{Les tâches restantes}
Il nous reste uniquement une semaine pour rédiger tout ces documents. Nous allons devoir mettre les bouché double pour pouvoir rendre tout les documents le vendredi 21 décembre. Nous sommes tout de même confiant sur la rédaction de tout ces documents étant donné que nous avons déjà rendu tous les autres projets, bien que la semaine à venir n'en soit pas moins chargée .

\end{document}
