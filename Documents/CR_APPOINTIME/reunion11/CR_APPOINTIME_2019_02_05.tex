\documentclass[a4paper]{article}
\usepackage[utf8]{inputenc}
\usepackage[T1]{fontenc}
\usepackage[french]{babel}
\usepackage[pdftex]{graphicx}
\usepackage{amsfonts}
\usepackage{tabularx}
\usepackage{float}
\usepackage{fancyhdr}

\fancyhead[L]{Entreprise : Pingouin}
\fancyhead[R]{Projet : Appointime}

\title{Compte-rendu de réunion du 05/02/2019}
\author{Didier Roche François}
\date{\today}

\begin{document}
\renewcommand{\contentsname}{Ordre du jour}
\maketitle
\thispagestyle{fancy}


\begin{itemize}
\item Lieu : Saint étienne
\item Heure de début :18h
\item Heure de fin : 17h20
\item Parcicipants : Gentile Pierre, Didier-Roche François
\end{itemize}

\bigbreak
\bigbreak
\bigbreak

\tableofcontents


\newpage

\section{Retour sur les travaux menés durant la semaine}
\subsection{Développement des pages connexion / inscription}
La connexion/inscription est maintenant totalement fonctionnelle. Ce module communique avec la base de donnés : nous pouvons récupérer des informations et en envoyer.


\subsection{Autres avancements}
Nous avons commencé à nous renseigner sur la conception des menus de navigation avec Flutter et nous avons commencé le formulaire permettant de renseigner les informations de l'entreprise d'un professionnel (forme générale).

\section{Tâches à venir}
\subsection{Menu de navigation et formulaire pour renseigner une entreprise}
Pour la semaine prochaine, Pierre devra finir le menu de navigation (faire en sorte qu'il soit intuitif, esthétique etc).
François devra finir le formulaire pour renseigner une entreprise avec toutes les validations (numéro siret unique, une entreprise par professionnel, communiation avec la base de données et estéthique).
\subsection{Esthétique de l'application}
Il faudra concevoir un logo et penser à l'esthétique générale de l'application afin de rendre cette dernière agréable à utiliser. Nous commencerons cela pour la semaine prochaine.


\end{document}
