\documentclass[a4paper]{article}
\usepackage[latin1]{inputenc}
\usepackage[T1]{fontenc}
\usepackage[french]{babel}
\usepackage[pdftex]{graphicx}
\usepackage{amsfonts}
\usepackage{tabularx}
\usepackage{float}
\usepackage{fancyhdr}

\fancyhead[L]{Entreprise : Pingouin}
\fancyhead[R]{Projet : Appointime}

\title{Compte-rendu de r�union du 12/10/2018}
\author{Gentile Pierre}
\date{\today}

\begin{document}
\renewcommand{\contentsname}{Ordre du jour}
\maketitle
\thispagestyle{fancy} 


\begin{itemize}
\item Lieu : Saint �tienne
\item Heure de d�but : 15h45
\item Heure de fin : 16h15
\item Parcicipants : Gentile Pierre, Didier-Roche Fran�ois
\end{itemize}

\bigbreak
\bigbreak
\bigbreak

\tableofcontents


\newpage

\section{Retour sur les travaux men�s durant la semaine}
\subsection{Choix de l'outil de d�veloppement}
Chaque membre du groupe a fais des recherches sur l'outil que nous
allons utiliser lors du developpement de l'application. Les deux
membres ayant d�ja une experience avec Android Studio, nous  nous
somme concentr� sur ionic durant la semaine et nous avons constat� que
cet outil n'apportait rien de plus qu'Android Studio si ce
n'est le developpement cross platforme. Nous avons en revanche appris
l'�xistence d'un autre outil (Flutter) utilisable sur Android Studio
et permetant le developpement cross platform.  

\subsection{R�daction de la base du cahier de sp�cifications}
Durant la semaine, nous somme parvenu � faire un premier squelette du
cahier de specificatiosn mais nous n'avons pas eu le temps de commencer
l'introduction.



\section{Les directives de la semaines � venir}
\subsection{Test de l'outil Flutter}
Chaque membre du groupe devra installer Android Studio et Flutter et
ce familiariser avec ces outils en essayant de produire une petite
application.
\subsection{R�daction du cahier de specifications}
\begin{itemize}

\item Fran�ois devra r�diger l'introduction du cahier de specifications.

\item Pierre devra r�flechir au contenu du cahier de specifications et �
l'organisation de ce dernier.




 
 
\end{document}