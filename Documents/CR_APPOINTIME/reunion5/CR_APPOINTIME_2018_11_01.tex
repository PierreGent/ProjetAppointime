\documentclass[a4paper]{article}
\usepackage[latin1]{inputenc}
\usepackage[T1]{fontenc}
\usepackage[french]{babel}
\usepackage[pdftex]{graphicx}
\usepackage{amsfonts}
\usepackage{tabularx}
\usepackage{float}
\usepackage{fancyhdr}

\fancyhead[L]{Entreprise : Pingouin}
\fancyhead[R]{Projet : Appointime}

\title{Compte-rendu de r�union du 01/11/2018}
\author{Gentile Pierre}
\date{\today}

\begin{document}
\renewcommand{\contentsname}{Ordre du jour}
\maketitle
\thispagestyle{fancy} 


\begin{itemize}
\item Lieu : Saint �tienne
\item Heure de d�but :20h
\item Heure de fin : 20h20
\item Parcicipants : Gentile Pierre, Didier-Roche Fran�ois
\end{itemize}

\bigbreak
\bigbreak
\bigbreak

\tableofcontents


\newpage

\section{Retour sur les travaux men�s durant la semaine}
\subsection{Test de l'outil Flutter}
Nous avons commencer � installer tout les outils n�cessaires �
l'utilisation de Flutter. Nous avons aussi regarder plusieurs
tutoriels pour avoir une id�e sur les difficult�es que nous pourions
rencontrer. 
\subsection{R�daction du cahier de specification}
Nous avons fais un premier squelette du cahier de sp�cification et
nous avons r�dig� l'introduction. 


\section{Les directives de la semaines � venir}
\subsection{Revoir la planification}
Notre plannification n'�tant pas claire, nous avons d�cid� de la
revoir compl�tement. Nous allons donc durant cette semaine devoir
refaire une nouvelle planification.
\subsection{Redaction du cahier de specification}
\begin{itemize}
\item Pierre devra r�diger la description g�n�rale. 
\item Fran�ois devras r�flechir � un squelette pour le document de
  test de recette.

 
 
\end{document}