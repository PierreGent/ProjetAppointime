\documentclass[a4paper]{article}
\usepackage[latin1]{inputenc}
\usepackage[T1]{fontenc}
\usepackage[french]{babel}
\usepackage[pdftex]{graphicx}
\usepackage{amsfonts}
\usepackage{tabularx}
\usepackage{float}
\usepackage{fancyhdr}

\fancyhead[L]{Entreprise : Pingouin}
\fancyhead[R]{Projet : Appointime}

\title{Compte-rendu de r�union du 08/11/2018}
\author{Didier Roche Fran�ois}
\date{\today}

\begin{document}
\renewcommand{\contentsname}{Ordre du jour}
\maketitle
\thispagestyle{fancy} 


\begin{itemize}
\item Lieu : Saint �tienne
\item Heure de d�but :13h00
\item Heure de fin : 13h25
\item Parcicipants : Gentile Pierre, Didier-Roche Fran�ois
\end{itemize}

\bigbreak
\bigbreak
\bigbreak

\tableofcontents


\newpage

\section{Retour sur les travaux men�s durant la semaine}
\subsection{R�vision de la planification}
Nous avons fait un nouveau diagramme de gantt plus ad�quat.
Certaines t�ches ont �t� red�coup�es, pr�cis�es et r�organis�es.
Le rouge et le jaune repr�sentent un seul membre du groupe tandis que
le orange repr�sente les deux.

\subsection{R�daction du cahier de sp�cification}
La description du document de recettes a �t� �ffectu�e.
Une approche pour le document de test de recette est avanc�e, elle vas
�tre mise au propre durant la semaine.

\section{Les directives de la semaines � venir}
\subsection{R�daction du cahier de sp�cification}
\begin{itemize}
\item Pierre devra commencer la r�daction du document des exigences
  sp�cifiques pour le cot� client. 
\item Fran�ois devra mettre au propre le squelette du document de test
  de recettes et commencer la r�daction du document des exigences
  sp�cifiques pour le cot� professionnel.

 
 
\end{document}