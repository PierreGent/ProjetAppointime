\documentclass[a4paper]{article}
\usepackage[utf8]{inputenc}
\usepackage[T1]{fontenc}
\usepackage[french]{babel}
\usepackage[pdftex]{graphicx}
\usepackage{amsfonts}
\usepackage{tabularx}
\usepackage{float}
\usepackage{fancyhdr}

\fancyhead[L]{Entreprise : Pingouin}
\fancyhead[R]{Projet : Appointime}

\title{Compte-rendu de réunion du 05/03/2019}
\author{Didier Roche François}
\date{\today}

\begin{document}
\renewcommand{\contentsname}{Ordre du jour}
\maketitle
\thispagestyle{fancy}


\begin{itemize}
\item Lieu : Saint étienne
\item Heure de début :17h15
\item Heure de fin : 17h30
\item Parcicipants : Didier-Roche François, Gentile Pierre
\end{itemize}

\bigbreak
\bigbreak
\bigbreak

\tableofcontents


\newpage

\section{Retour sur les travaux menés durant la semaine}
\subsection{Barre de recherche}
François à ajouté une barre de recherche permettant de rechercher une entreprise par nom, par adresse ou par mot clefs dans la description.

\subsection{Autres tâches}
Pierre à lissé l'esthétique de l'application, tout est plus uniforme et les modules s'integrent mieux entre eux.
Pour ce qui est des taches que l'on devait aborder cette semaine, nous avons pris du retard car nous nous penchons sur les projet ayant une date de rendu moins tardive. En effet, ce projet est très prenant et nous avons eu tendance à le privilégier aux autres, nous devons maintenant rattraper notre retard.

\section{Tâches à venir}
\subsection{Horaires et disponibilités}
Pour la semaine prochaine, pierre devra créer un formulaire permettant à un professionnel de renseigner les horaires de son entreprise.
\subsection{Prise de rendez vous}
François devra integrer la prise de rendez-vous en fonction des horaires et des disponibilités des entreprises.

\end{document}
