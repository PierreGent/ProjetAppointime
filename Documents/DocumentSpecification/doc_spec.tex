\documentclass{article}
\usepackage[utf8]{inputenc}
\usepackage[T1]{fontenc}
\usepackage[french]{babel}
\usepackage[bookmarks=true]{hyperref}

\author{Gentile Pierre, Didier-Roche François}
\date{\today}
\title{Document de spécification des exigences}


\begin{document}

\maketitle

\newpage
\tableofcontents


\section{Introduction}
\subsection{Objet}
Appointime cherche à faciliter la prise de rendez vous et l’acces aux disponnibilités des artisants/entreprises de services,
Elle permet de mettre en relation des particuliers et des  artisants/entreprises de services intuitivement et rapidement.
\subsection{Portée du projet}
Appointime permettra la mise en relation et la prise de rendez vous rapide pour des taches simples/de routine entre une entreprise et un client,
L’application sera donc divisée en deux: partie particulier et partie professionelle, Le professionnel pourra indiquer ses disponnibilités et precisera son domaine de competance ainsi que les taches de base qu’il peut effectuer,
Le client quant a lui pourra faire des recherches selon ses besoins et trouver rapidement et intuitvement des professionnels disponibles aux alentours
\subsection{Définitions, acronymes, abréviations}

\subsection{Références}

\subsection{Vue d ensemble}


\section{Description générale}
\subsection{Environnement}
\begin{itemize}
\item L'application décrite est une application mobile sur IOS et
Android.
\item L'application est dépendante des systémes d'exploitation que
nous utiliserons pour la gestion des notifications.
\item L'application est adaptée pour n'importe quel smartphone ayant IOS ou Android comme
systeme d'exploitation.
\item Les actions de l'utilisateur s'éffectue via l'écran
  tactile du smartphone.
\item L'application communique avec une base de donnée distante
  via une connexion 4G ou Wifi.
\end{itemize}
\subsection{Fonctions}
\begin{itemize}
\item L'application permet à n'importe quel professionnel de créer un
calendrier de rendez-vous personnalisé.
\item L'application permet à n'importe quelle personne de rechercher un
  professionnel près de cher lui.
\item L'application permet à n'importe quelle personne de prendre un
  rendez-vous cher un professionnel selon de calendrier paramétré par
  ce dernier.
\end{itemize}
\subsection{Caractéristiques des utilisateurs}
\begin{itemize}
\item L'utilisateur peut étre soit un professionnel de n'importe soit
  un client.
\item L'utilisateur doit savoir utiliser un smartphone convenablement.
\item Le client n'a besoin d'aucune connaissance technique.
\item Le professionnel doit avoir des connaissance dans son domaine
  d'activité.
\end{itemize}
\subsection{Contraintes}
\begin{itemize}
\item La prise et l'annulation de rendez-vous doit ce plier à la
  politique de chaque professionnel.
\item Les informations des utilisateurs devront uniquement être
  utilisé dans le cadre de l'application.
\end{itemize}
\subsection{Hypothèses et dépendances}


\section{Exigences spécifiques}
\subsection{Exigences des interfaces externes}
\subsubsection{Interfaces avec les utilisateur}
\paragraph{Les information des utilisateurs}
\begin{itemize}
 \item Un formulaire d'inscription doit permettre à l'utilisateur de
   s'inscrire en entrant différents champs textuels (email, mot de
   passe, nom, numéro de téléphone, adresse) nécéssaires au
   système et en appuyant sur
   un bouton \og S'inscrire \fg{}.
\item Un formulaire doit permettre à un utilisateur de modifier les
  informations mentionnée durant son inscription.
\item Un formulaire de connexion doit permetre à un utilisateur de
  s'identifier pour avoir accès aux services du système en entrant
  différents champs textuels (email, mot de passe) et en appuyant sur
  un bouton \og Se connecter \fg{}.
\item Une reconnaissace par empreinte digitale devrait permette à un
  utilisateur de pouvoir s'identifier rapidement dans le cas ou cette
  fonctionnalitée est diponible sur le téléphone.
\item Une reconnaissance faciale devrait permette à un
  utilisateur de pouvoir s'identifier rapidement dans le cas ou cette
  fonctionnalitée est diponible sur le téléphone.
\end{itemize}
\paragraph{Préférences des utilisateurs}
\begin{itemize}
\item Le client doit pouvoir acceder au menu \og Paramètres \fg{}. de l'application en
  appuyant sur un bouton à l'écran.
\item Le client doit pouvoir activer ou désactiver les notifications
  de l'application dans le menu paramètre.
\item Le client devrait pouvoir modifier le son des notification
  (volume et tonalitée).
\item Le client doit pouvoir activer ou désactiver la localisation
  dans le menu paramètres.
\item Le client devrait pouvoir activer le thème qui modifirais les
  couleurs de l'application. 
\end{itemize}
\paragraph{Recherche et prise de rendez-vous cher un professionnel}
\begin{itemize}
\item Un champs textuel doit permetre à l'utilisateur d'éfectuer une
  recherche par profession ou directement par nom
  d'entreprise.
\item Après avoir fais une recherche, le client doit pouvoir selectionner un
  professionnel dans la liste affichée. 
\item La selection d'un
  professionnel de la liste doit permetre d'afficher le
  calendrier du professionnel en question.
\item Le client doit pouvoir selectionner une prestation parmis une liste
  proposée par le professionnel puis il peut reserver un crénaux
  horaire pour cette prestation en appuyant sur les crénaux
  disponibles affichés sur le calendrier.
\item Le client doit pouvoir ajouter un professionnel en favoris gràce
  a un bouton présent sur le calendrier d'un professionnel et sur la
  liste déroulante de recherche.
\end{itemize}


\paragraph{Paramétrage du calendrier côté profesionnel}
\begin{itemize}
\item Un menu doit permettre au professionnel d'ajouter une nouvelle
  prestation en appuyant sur un bouton \og Ajouter une prestation \fg{}, de
  supprimer une prestation en appuyant sur un bouton \og Supprimer une
  prestation \fg{} dans la liste de ses prestation déjà existantes, de modifier une prestation en appuyant sur un
  bouton \og Modifier une prestation \fg{} dans la liste de ses
  prestations déjà existantes
  ou de modifier son calendrier. 
\item Un formulaire doit permettre de créer une prestation. Ce
  formulaire doit contenir les champs textuels suivant \og Nom \fg{},
  \og Durée \fg{} et \og Prix \fg{}.  
\item Un formulaire doit permettre d'éffectuer des modifications sur
  une prestation. Ce formulaire doit contenir un champ nom, un champ
  durée et un champ prix.  
\item Un calendrier modifiable doit permetre à l'utilisateur de
  modifier ses diponibilitées en modifiant les jours et les horaires
  auquelles il peux effectuer ses prestations.
\item Le calendrier modifiable doit pemettre à l'utilisateur de
  visualiser et de gérer les rendez vous déjà pris. Sur un crénaux
  pris, il doit afficher les informations de l'utilisateur qui a pris
  le rendez-vous. Il doit aussi permettre au professionnel de valider
  ou non les rendez vous. 
\end{itemize}

\subsection{Exigences fonctionnelles}
\paragraph{Pour la connexion le systeme doit : }

\begin{itemize}
\item Verifier si l'email saisi existe dans la base de données
\begin{itemize}
\item Si il existe le systeme doit verifier si le mot de passe saisi
   correspond à l'email saisi.
\begin{itemize}
     \item Si il correspond, le systeme doit autoriser la connexion et
     rediriger l'utilisateur a l'acceuil. Ce dernier aura accès à son
     profile ainsi qu'au pages de reservations/gestion de calendrier
     si c'est un particulier/professionnel.
    \item Si il ne correspond pas, le systeme doit afficher une erreur
      et renvoyer sur la page de connexion
\end{itemize}
  \item Si il n'existe pas le systeme doit renvoyer une erreur.
\end{itemize}
\end{itemize}

\paragraph{Pour l'inscription le systeme doit : }
\begin{itemize}
\item Verifier si les données saisies sont valides
\begin{itemize}
\item Si les données sont valides, le systeme doit verifier si le mail
  est deja existant.
\begin{itemize}
     \item Si les données n'existent pas encore dans la bdd le systeme
       doit les enregister et rediriger vers la page de connexion
       ainsi qu'indiquer le bon deroulement de l'opération via un message.
    \item Si les données existent deja le systeme doit rediriger vers
      la page de connexion et indiquer que le mail est deja utilisé
      via un message.
\end{itemize}
  \item Si les données ne sont pas valides le systeme doit l'indiquer
    via un message detaillant les erreurs.
\end{itemize}
\end{itemize}




\section{Annexes}


\section{Index}


\end{document}
