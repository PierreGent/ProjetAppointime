\documentclass{article}
\usepackage[utf8]{inputenc}
\usepackage[T1]{fontenc}
\usepackage[frenchb]{babel}
\usepackage[bookmarks=true]{hyperref}
\usepackage{lmodern}
\usepackage{fullpage}

\author{Gentile Pierre, Didier--Roche François}
\date{\today}
\title{Cahier de test d'intégration}


\begin{document}

\maketitle

\begin{abstract}

  Ce document recense les tests d'intégration qui doivent être menés lors de la phase de test de l'application Appointime. 
 Ce document est représenté à l'aide de tableaux avec :
   le nom de la partie, les tests à effectuer, les résultats attendus et les résultats observés.

\end{abstract}

\newpage
\begin{center}
  \begin{tabular}{|p{4cm}|p{4cm}|p{4cm}|p{4cm}|}
    \hline
    \multicolumn{4}{|c|}{\textbf{Tests d'intégration}} \\
    \hline
    \textbf{Test effectué} & \textbf{Résultat attendu} & \textbf{Résultat observé} & \textbf{Correction apportée}\\
    \hline
    
  

     Si des résultats sont présents après une recherche, j'appuie sur n'importe quel lien dans la liste&
      Le système me redirige vers la page dédiée à l'entreprise correspondant au lien sur lequel j'ai cliqué &
      ...&
      ... \\
       \hline
      \hline


      Si des résultats sont présent après une recherche, j'appuie sur le bouton \og Ajouter aux favoris \fg{} dans la liste des professionnels&
      Le système ajoute l'information dans la base de données et sur la page d'affichage des favoris et le bouton \og Ajouter aux favoris \fg{} deviendra \og Supprimer des favoris \fg{}&
      ...&
      ... \\
      
      
      \hline
      \hline


     Je me connecte avec succès et suis redirigé sur la page d'accueil&
      Mes favoris sont présents dans la page d'accueil&
      ...&
      ... \\
      
      \hline
      \hline


     Je suis connecté&
      Les liens du menu me redirigent tous sur les sections correspondantes et les informations indiquées me concernent bien  &
      ...&
      ... \\
      
       \hline
      \hline

      Etant sur la page de confirmation, je confirme ma réservation&
      Une notification est envoyée au professionnel concerné, dans la section \og Réservations à confirmer\fg{} il peut voir la demande. Le particulier
      verra cette demande dans la section \og En attente de confirmation\fg{} Le créneaux n'est plus disponible pour les autres particuliers&
      ...&
      ... \\
  \hline
      \hline
      Etant un professionnel, en recevant la demande de réservation je l'accepte/elle est accéptée automatiquement&
      Une notification est envoyée au particulier qui a fait sa demande en lui indiquant que le rendez-vous a été accepté
      les détails de cette réservation sont disponibles dans la section \og Mes rendez-vous à venir\fg{} du particulier, ne le sont plus dans la section 
      \og En attente de confirmation\fg{} et ne le sont plus dans la section \og Rendez-vous à confirmer\fg{} du professionnel&
      ...&
      ... \\
        \hline
       

    \end{tabular}
  \end{center}
      \newpage
\begin{center}
  \begin{tabular}{|p{4cm}|p{4cm}|p{4cm}|p{4cm}|}
    \hline
    \multicolumn{4}{|c|}{\textbf{Tests d'integration}} \\
    \hline
    \textbf{Test éffectué} & \textbf{Résultat attendu} & \textbf{Résultat observé} & \textbf{Correction apportée}\\
    \hline
      Etant un professionnel, en recevant la demande de réservation je la refuse&
      Une notification est envoyée au particulier qui a fait sa demande que le rendez-vous à été refusé. Le créneau est de nouveau disponible pour les autres particuliers&

      ...&
      ... \\
      
       \hline
      \hline
       Etant un professionnel, je signale un particulier&
      Le particulier perds un point de crédibilité&

      ...&
      ... \\
       \hline
      
      \hline
      \hline
      La notification par sms est activée et je recois une notification de prise de rendez-vous&
      Le smartphone reçoit un sms&

      ...&
      ... \\
       \hline
       

    \end{tabular}
  \end{center}
    \end{document}
