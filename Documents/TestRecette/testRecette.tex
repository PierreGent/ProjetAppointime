\documentclass{article}
\usepackage[utf8]{inputenc}
\usepackage[T1]{fontenc}
\usepackage[frenchb]{babel}
\usepackage[bookmarks=true]{hyperref}
\usepackage{lmodern}

\author{Gentile Pierre}
\date{\today}
\title{Cahier de test de recette}


\begin{document}

\maketitle

\begin{abstract}
<<<<<<< HEAD
  Ce documents recense les test unitaires qui doivents être mené lors de la phase de test de l'application Appointime. Les grandes parties de l'application seront représentées dans ce document a l'aide d'un tableau avec : le nom de la partie, les tests a effectuer, les résultats attendus et les résultats observés.
=======
  Ce document recense les test unitaires qui doivents être menés lors de la phase de test de l'application Appointime. Les grandes parties de l'application seront représentées dans ce document à l'aide d'un tableau avec : le nom de la partie, les tests à effectuer, les résultats attendus et les résultats observés.
>>>>>>> 63e10de7ad9a5ea5d900f0df9458a8cdeca5ef6d
\end{abstract}

\newpage
\begin{center}
  \begin{tabular}{|p{5cm}|p{5cm}|p{5cm}|}
    \hline
    \multicolumn{3}{|c|}{\textbf{Informations des utilisateur (Inscription)}} \\
    \hline
    \textbf{Test éffectué} & \textbf{Résultat attendu} & \textbf{Résultat observé} \\
    \hline

    J'entre des informations valides, l'email n'est pas déja présent dans la base de données et j'appuie sur le bouton \og s'inscrire \fg{} &
    \begin{itemize}
<<<<<<< HEAD
      \item Le système me redirique vers la page de connexion en m'indiquant que j'ai bien été enregistré.
      \item Les informations ont bien été inseré dans la base de données
=======
      \item Le système me redirige vers la page de connexion en m'indiquant que j'ai bien été enregistré.
      \item Les informations ont bien été inserées dans la base de données
>>>>>>> 63e10de7ad9a5ea5d900f0df9458a8cdeca5ef6d
      \end{itemize}&
      ... \\

      \hline
      \hline
      J'entre des informations valides et l'email est déja présent dans la base de données et j'appuie sur le bouton \og s'inscrire \fg{} &
<<<<<<< HEAD
      Le système me redirique vers la page d'inscription en m'indiquant que l'email est déjà utilisé &
=======
      Le système me redirige vers la page d'inscription en m'indiquant que l'email est déjà utilisé &
>>>>>>> 63e10de7ad9a5ea5d900f0df9458a8cdeca5ef6d
      ... \\


      \hline
      \hline
      J'entre une ou plusieurs informations invalides et j'appuie sur le bouton \og s'inscrire \fg{} &
<<<<<<< HEAD
      Le système me redirique vers la page d'inscription en m'indiquant les données invalides&
=======
      Le système me redirige vers la page d'inscription en m'indiquant les données invalides&
>>>>>>> 63e10de7ad9a5ea5d900f0df9458a8cdeca5ef6d
      ... \\
      \hline

    \end{tabular}
  \end{center}


  \begin{center}
    \begin{tabular}{|p{5cm}|p{5cm}|p{5cm}|}
      \hline
<<<<<<< HEAD
      \multicolumn{3}{|c|}{\textbf{Informations des utilisateur (Connexion)}} \\
=======
      \multicolumn{3}{|c|}{\textbf{Informations des utilisateurs (Connexion)}} \\
>>>>>>> 63e10de7ad9a5ea5d900f0df9458a8cdeca5ef6d
      \hline
      \textbf{Test éffectué} & \textbf{Résultat attendu} & \textbf{Résultat observé} \\
      \hline

      J'entre des informations valides, la combinaison email+mot de passe est présente dans la base de données et j'appuie sur le bouton\og se connecter \fg{}.&
<<<<<<< HEAD
      Le système me redirige vers la page d'acceuil.&
=======
      Le système me redirige vers la page d'accueil.&
>>>>>>> 63e10de7ad9a5ea5d900f0df9458a8cdeca5ef6d
      ... \\

      \hline
      \hline
      J'entre des informations invalides et/ou la combinaison email+mot de passe n'est pas présente dans la base de données et j'appuie sur le bouton \og se connecter \fg{}. &
<<<<<<< HEAD
      Le système me redirique vers la page de connexion en m'indiquant une erreur.&
=======
      Le système me redirige vers la page de connexion en m'indiquant une erreur.&
>>>>>>> 63e10de7ad9a5ea5d900f0df9458a8cdeca5ef6d
      ... \\

      \hline
      \hline
      Pour les smartphones equipés, sur la page de connexion, je pose mon doigt sur le lecteur d'empreinte digitale alors que l'empreinte est enregistrée sur le smartphone&
<<<<<<< HEAD
      Le système me redirique vers la page de d'acceuil.&
=======
      Le système me redirige vers la page de d'accueil.&
>>>>>>> 63e10de7ad9a5ea5d900f0df9458a8cdeca5ef6d
      ... \\

      \hline
      \hline
      Pour les smartphones equipés, sur la page de connexion, je pose mon doigt sur le lecteur d'empreinte digitale alors que l'empreinte n'est pas enregistrée sur le smartphone&
<<<<<<< HEAD
      Le système me redirique vers la page de connexion en m'indiquant une erreur.&
=======
      Le système me redirige vers la page de connexion en m'indiquant une erreur.&
>>>>>>> 63e10de7ad9a5ea5d900f0df9458a8cdeca5ef6d
      ... \\

      \hline
      \hline
      Pour les smartphones equipés, sur la page de connexion, j'utilise la reconnaissance faciale alors que mon visage est enregistré sur le smartphone&
<<<<<<< HEAD
      Le système me redirique vers la page d'acceuil&
=======
      Le système me redirige vers la page d'accueil&
>>>>>>> 63e10de7ad9a5ea5d900f0df9458a8cdeca5ef6d
      ... \\

      \hline
      \hline
      Pour les smartphones equipés, sur la page de connexion, j'utilise la reconnaissance faciale alors que mon visage n'est pas enregistré sur le smartphone&
<<<<<<< HEAD
      Le système me redirique vers la page de connexion en m'indiquant une erreur.&
=======
      Le système me redirige vers la page de connexion en m'indiquant une erreur.&
>>>>>>> 63e10de7ad9a5ea5d900f0df9458a8cdeca5ef6d
      ... \\


      \hline

    \end{tabular}
  \end{center}


  \begin{center}
    \begin{tabular}{|p{5cm}|p{5cm}|p{5cm}|}
      \hline
<<<<<<< HEAD
      \multicolumn{3}{|c|}{\textbf{Informations des utilisateur (Modification des informations)}} \\
=======
      \multicolumn{3}{|c|}{\textbf{Informations des utilisateurs (Modification des informations)}} \\
>>>>>>> 63e10de7ad9a5ea5d900f0df9458a8cdeca5ef6d
      \hline
      \textbf{Test éffectué} & \textbf{Résultat attendu} & \textbf{Résultat observé} \\
      \hline

<<<<<<< HEAD
      Je modifie une ou plusieurs informations en entrant des information valides. Si je modifie l'email, le nouvel email n'est pas déjà etre présent dans la base de données et j'appuie sur le bouton \og modifier \fg{}.&
=======
      Je modifie une ou plusieurs informations en entrant des information valides. Si je modifie l'email et que le nouvel email n'est pas déjà présent dans la base de données et j'appuie sur le bouton \og modifier \fg{}.&
>>>>>>> 63e10de7ad9a5ea5d900f0df9458a8cdeca5ef6d
      Le systeme me redirige vers la page d'acceuil en m'indiquant que mes information ont bien été modifiées. &
      ... \\

      \hline
      \hline
      J'entre des information valides dont l'email mais le nouvel email est déjà present dans la base de données et j'appuie sur le bouton \og modifier \fg{} &
<<<<<<< HEAD
      Le système de redirige vers la page de modification des informations en m'indiquant que l'email existe déjà &
=======
      Le système me redirige vers la page de modification des informations en m'indiquant que l'email existe déjà &
>>>>>>> 63e10de7ad9a5ea5d900f0df9458a8cdeca5ef6d
      ... \\

      \hline
      \hline
      J'entre des information invalides.&
<<<<<<< HEAD
      Le système de redirige vers la page de modification des informations en m'indiquant les champs invalides.&
=======
      Le système me redirige vers la page de modification des informations en m'indiquant les champs invalides.&
>>>>>>> 63e10de7ad9a5ea5d900f0df9458a8cdeca5ef6d
      ... \\

      \hline

    \end{tabular}
  \end{center}


  \begin{center}
    \begin{tabular}{|p{5cm}|p{5cm}|p{5cm}|}
      \hline
      \multicolumn{3}{|c|}{\textbf{Préférences des utilisateurs}} \\
      \hline
      \textbf{Test éffectué} & \textbf{Résultat attendu} & \textbf{Résultat observé} \\
      \hline

<<<<<<< HEAD
      En réglages basiques (notification activée), je recoit une notification&
      Le smartphone m'affiche la notification dans le menu du smartphone et emmet un son&
=======
      En réglages basiques (notification activée), je recois une notification&
      Le smartphone m'affiche la notification dans le menu du smartphone et emmet un son et/ou une vibration&
>>>>>>> 63e10de7ad9a5ea5d900f0df9458a8cdeca5ef6d
      ... \\

      \hline
      \hline
      Je désactive le notification en appuyant sur le bouton et je recoit une notification&
      Le smartphone n'affiche aucune notification et n'émet aucun son&
      ... \\

      \hline
      \hline
      Je réactive les notifications et je recoit une notification&
<<<<<<< HEAD
      Le smartphone m'affiche la notification dans le menu du smartphone et emmet un son&
=======
      Le smartphone m'affiche la notification dans le menu du smartphone et emmet un son et/ou une vibration&
>>>>>>> 63e10de7ad9a5ea5d900f0df9458a8cdeca5ef6d
      ... \\

      \hline
      \hline
      Les notifications étant activées, je regle le son des notification et je recoit une notification&
      Le smartphone m'affiche la notification dans le menu du smartphone et emmet un son qui sera plus ou moins fort selon la modification que j'ai éffectué&
      ... \\

      \hline
      \hline
<<<<<<< HEAD
      Les notifications étant activées, je modifie la choisis une nouvelle tonalité de notifications et je recoit une notification&
      Le smartphone m'affiche la notification dans le menu du smartphone et emmet un son qui sera plus ou moins fort selon la modification que j'ai éffectué&
=======
      Les notifications étant activées, je choisis une nouvelle tonalité de notifications et je recoit une notification&
      Le smartphone m'affiche la notification dans le menu du smartphone et emmet un son différent selon la modification que j'ai éffectué&
>>>>>>> 63e10de7ad9a5ea5d900f0df9458a8cdeca5ef6d
      ... \\

      \hline
      \hline
      Etant dans le thème de base, j'active de thème nuit en appuyant sur le bouton prévu a cet effet.&
      Les couleurs de l'application changent quelque soit la page.&
      ... \\



      \hline

    \end{tabular}
  \end{center}

  \begin{center}
    \begin{tabular}{|p{5cm}|p{5cm}|p{5cm}|}
      \hline
      \multicolumn{3}{|c|}{\textbf{Recherche d'un professionnel}} \\
      \hline
      \textbf{Test éffectué} & \textbf{Résultat attendu} & \textbf{Résultat observé} \\
      \hline

      J'entre un nom d'entreprise et ce dernier est présent dans la base de données&
      Le système m'affiche le lien menant vers le calendrier de cette entreprise&
      ... \\

      \hline
      \hline
      J'entre un domaine d'activité présent dans la base de données&
      le système m'affiche une liste de liens vers le calendrier des professionnel dans ce domaine d'activitée&
      ... \\

      \hline
      \hline
      J'utilise mon appareil photo pour annalyser un flash code qui est présent dans la base de données&
      Le système affiche le lien du professionnel correspondant au flashcode&
      ... \\

      \hline
      \hline
      J'entre un nom d'entreprise ou un domaine d'activitées qui n'est pas présent dans la base de données&
      Le système m'indique qu'aucun résultat n'a été trouvé&
      ... \\

      \hline
      \hline
      J'utilise mon appareil photo pour annalyser un flash code qui n'est pas présent dans la base de données&
      Le système m'indique qu'aucun résultat n'à été trouvé &
      ... \\

      \hline
      \hline

<<<<<<< HEAD
      Si une recherche c'est bien déroulée, j'appuie le bouton \og Ajouter aux favoris \fg{} dans la liste des professionnels&
=======
      Si une recherche s'est bien déroulée, j'appuie le bouton \og Ajouter aux favoris \fg{} dans la liste des professionnels&
>>>>>>> 63e10de7ad9a5ea5d900f0df9458a8cdeca5ef6d
      Le système ajoute l'information dans la base de donnée et sur la page favori, on peut voir le professionnel ajouté&
      ... \\

      \hline
      \hline

<<<<<<< HEAD
      Si une recherche c'est bien déroulée, j'appuie n'importe quel lien de professionnel dans la liste&
=======
      Si une recherche s'est bien déroulée, j'appuie n'importe quel lien de professionnel dans la liste&
>>>>>>> 63e10de7ad9a5ea5d900f0df9458a8cdeca5ef6d
      Le système me redirige vers la page de calendrier du professionnel correspondant au lien sur lequel j'ai cliqué &
      ... \\



      \hline

    \end{tabular}
  \end{center}


  \begin{center}
    \begin{tabular}{|p{5cm}|p{5cm}|p{5cm}|}
      \hline
<<<<<<< HEAD
      \multicolumn{3}{|c|}{\textbf{Prise de rendez-vous cher un professionnel}} \\
=======
      \multicolumn{3}{|c|}{\textbf{Prise de rendez-vous chez un professionnel}} \\
>>>>>>> 63e10de7ad9a5ea5d900f0df9458a8cdeca5ef6d
      \hline
      \textbf{Test éffectué} & \textbf{Résultat attendu} & \textbf{Résultat observé} \\
      \hline

      Je séléctionne une prestation dans le menu déroulant de la page du calendrier d'un professionnel&
      Le système m'affiche le calendrier du professionnel avec les crénaux disponibles.&
      ... \\

      \hline
      \hline

<<<<<<< HEAD
      Une fois le calendrier affiché, j'appuie sur un crénaux disponible&
=======
      Une fois le calendrier affiché, j'appuie sur un crénau disponible&
>>>>>>> 63e10de7ad9a5ea5d900f0df9458a8cdeca5ef6d
      Le système me redirige vers la page de confirmation&
      ... \\

      \hline
      \hline

<<<<<<< HEAD
      Etant sur la page de confirmation, je confirme ma reservation&
=======
      Etant sur la page de confirmation, je confirme ma réservation&
>>>>>>> 63e10de7ad9a5ea5d900f0df9458a8cdeca5ef6d
      Une notification est envoyée au professionnel concerné, dans la section \og Reservation à confirmer\fg{} on peut voir la demande. Le crénaux n'est plus disponible pour les autres particuliers&
      ... \\

      \hline
      \hline
      Etant sur la page de confirmation,j'annule ma reservation&
      Le système me redirige vers la page précédente&
      ... \\

      \hline
      \hline
<<<<<<< HEAD
      Etant un professionnel, en recevant la demande de réservation je l'accepte&
      Une notification est envoyée au particulier qui a fais sa demande en lui indiquant que le rendez-vous a été accepté&
=======
      Etant un professionnel, en recevant la demande de réservation je l'accepte/elle est accéptée automatiquement&
      Une notification est envoyée au particulier qui a fait sa demande en lui indiquant que le rendez-vous a été accepté&
>>>>>>> 63e10de7ad9a5ea5d900f0df9458a8cdeca5ef6d
      ... \\

      \hline
      \hline
      Etant un professionnel, en recevant la demande de réservation je la refuse&
<<<<<<< HEAD
      Une notification est envoyée au particulier qui a fais sa demande que le rendez-vous à été refusé. Le crénaux est de nouveaux disponible pour les autres particuliers&
=======
      Une notification est envoyée au particulier qui a fait sa demande que le rendez-vous à été refusé. Le crénaux est de nouveaux disponible pour les autres particuliers&
>>>>>>> 63e10de7ad9a5ea5d900f0df9458a8cdeca5ef6d
      ... \\

      \hline

    \end{tabular}
  \end{center}

  \begin{center}
    \begin{tabular}{|p{5cm}|p{5cm}|p{5cm}|}
      \hline
<<<<<<< HEAD
      \multicolumn{3}{|c|}{\textbf{Parametrage du calendrier d'un professionnel}} \\
=======
      \multicolumn{3}{|c|}{\textbf{Parametrage des prestations d'un professionnel}} \\
>>>>>>> 63e10de7ad9a5ea5d900f0df9458a8cdeca5ef6d
      \hline
      \textbf{Test éffectué} & \textbf{Résultat attendu} & \textbf{Résultat observé} \\
      \hline

      Dans le menu de paramétrage du calendrier j'appuie sur le bouton \og ajouter une prestation \fg{}&
      Le système me redirige vers le formulaire d'ajout de prestation&
      ... \\

      \hline
      \hline

      Dans le formulaire d'ajout de prestation, j'entre des informations
      valides et j'appuie sur le boutton \og Valider \fg{}&
      Le système ajoute les information en base de données et me redirige
      vers le menu
      de parametrage
<<<<<<< HEAD
      du calendrier&
=======
      du prestations&
>>>>>>> 63e10de7ad9a5ea5d900f0df9458a8cdeca5ef6d
      ... \\

      \hline
      \hline

      Dans le formulaire d'ajout de prestation j'entre des informations
      invalides et j'appuie sur \og Valider \fg{} &
      Le système m'indique les erreurs que j'ai fait et me renvoie sur le
      formulaire d'ajout de prestation &
      ... \\

      \hline
      \hline
      Dans le menu de parametrage du calendrier j'appuie sur le bouton \og
      Modifier \fg{} de n'importe quelle prestation&
      Le système me redirige vers le formulaire de modification de prestation&
      ... \\

      \hline
      \hline
      Dans le formulaire de modification de prestation, j'entre des
      informations valides et j'appuie sur le boutton \og Modifier \fg{}&
<<<<<<< HEAD
      Le système modifie les information en base de donnée et me redirique
      vers
      le
      menu de
      parametrage
      du
      calendrier.&
=======
      Le système modifie les information en base de donnée et me redirige
      vers
      le formulaire de modification de prestation.&
>>>>>>> 63e10de7ad9a5ea5d900f0df9458a8cdeca5ef6d
      ... \\

      \hline
      \hline
      Dans le formulaire d'ajout de prestation, j'entre des informations
<<<<<<< HEAD
      valides et j'appuie sur le boutton \og Valider \fg{}&
=======
      invalides et j'appuie sur le boutton \og Valider \fg{}&
>>>>>>> 63e10de7ad9a5ea5d900f0df9458a8cdeca5ef6d
      Le système m'indique les erreurs que j'ai fait et me renvoie sur le
      formulaire de modification de prestation&
      ... \\

      \hline
      \hline
      Dans le menu de parametrage du calendrier j'appuie sur le bouton \og
      Supprimer \fg{} de n'importe quelle prestation&
<<<<<<< HEAD
      Le système suprime la prestion en base de donnée et je ne la vois plus
=======
      Le système supprime la prestation en base de donnée et je ne la vois plus
>>>>>>> 63e10de7ad9a5ea5d900f0df9458a8cdeca5ef6d
      apparaitre dans
      la liste. &
      ... \\



      \hline

    \end{tabular}
  \end{center}

<<<<<<< HEAD
  \begin{center}
    \begin{tabular}{|p{5cm}|p{5cm}|p{5cm}|}
      \hline
      \multicolumn{3}{|c|}{\textbf{Signalement d'un particulier}} \\
=======





\begin{center}
    \begin{tabular}{|p{5cm}|p{5cm}|p{5cm}|}
      \hline
      \multicolumn{3}{|c|}{\textbf{Parametrage de l'entreprise d'un professionnel}} \\
      \hline
      \textbf{Test éffectué} & \textbf{Résultat attendu} & \textbf{Résultat observé} \\
      \hline

      Dans le menu de paramétrage d'une entreprise j'appuie sur le bouton \og modifier les informations \fg{}&
      Le système me redirige vers le formulaire de gestion des informations sur l'entreprise&
      ... \\

      \hline
      \hline

      Dans le formulaire de gestion des informations sur l'entreprise, j'entre des informations
      valides et j'appuie sur le boutton \og Valider \fg{}&
      Le système ajoute les information en base de données et me redirige
      vers le menu d'affichage des informations sur l'entreprise&
      ... \\

      \hline
      \hline

      Dans le formulaire de gestion des informations sur l'entreprise, j'entre des informations
      invalides et j'appuie sur \og Valider \fg{} &
      Le système m'indique les erreurs que j'ai fait et me renvoie sur le
      formulaire  de gestion des informations sur l'entreprise&
      ... \\

   


      \hline

    \end{tabular}
  \end{center}







\begin{center}
    \begin{tabular}{|p{5cm}|p{5cm}|p{5cm}|}
      \hline
      \multicolumn{3}{|c|}{\textbf{Parametrage du calendrier d'un professionnel}} \\
      \hline
      \textbf{Test éffectué} & \textbf{Résultat attendu} & \textbf{Résultat observé} \\
      \hline

      Dans le menu de paramétrage du calendrier j'appuie sur le bouton \og parametrer les horaires d'ouverture \fg{}&
      Le système me redirige vers le formulaire de gestion des horaires d'ouverture&
      ... \\

      \hline
      \hline

      Dans le formulaire de gestion des horaires d'ouverture, j'entre des informations
      valides et j'appuie sur le boutton \og Valider \fg{}&
      Le système ajoute les information en base de données et me redirige
      vers le menu de gestion des horaires d'ouverture&
      ... \\

      \hline
      \hline

      Dans le formulaire de gestion des horaires d'ouverture, j'entre des informations
      invalides et j'appuie sur \og Valider \fg{} &
      Le système m'indique les erreurs que j'ai fait et me renvoie sur le
      formulaire de gestion des horaires d'ouverture,&
      ... \\

      \hline
      \hline
      Dans le menu de parametrage du calendrier j'appuie sur le bouton \og
      Modifier \fg{} de n'importe quel jour&
      Le système me redirige vers le formulaire de modification des horaires pour ce jour&
      ... \\

      \hline
      \hline
      Dans le formulaire de modification d'horaires d'ouverture du jour, j'entre des
      informations valides et j'appuie sur le boutton \og Modifier \fg{}&
      Le système modifie les information en base de donnée et me redirige
      vers
      le
      menu de gestion des horaires d'ouverture.&
      ... \\

      \hline
      \hline
      Dans le formulaire de gestion des horaires d'ouverture du jour, j'entre des informations
      invalides et j'appuie sur le boutton \og Valider \fg{}&
      Le système m'indique les erreurs que j'ai fait et me renvoie sur le
      formulaire de modifications d'horaires du jour &
      ... \\

    


      \hline

    \end{tabular}
  \end{center}













  \begin{center}
    \begin{tabular}{|p{5cm}|p{5cm}|p{5cm}|}
      \hline
      \multicolumn{3}{|c|}{\textbf{Rappels de rendez-vous}} \\
>>>>>>> 63e10de7ad9a5ea5d900f0df9458a8cdeca5ef6d
      \hline
      \textbf{Test éffectué} & \textbf{Résultat attendu} & \textbf{Résultat observé} \\
      \hline

      Étant un particulier, je paramètre un ou plusieurs rappel de rendez-vous&
      Je recoit une notification à l'heure que j'ai paramétré selon l'heure de mon telephone
      et le rappel est bien enregistré dans la base de donnée locale&
      ... \\


      \hline

    \end{tabular}
  \end{center}

  \begin{center}
    \begin{tabular}{|p{5cm}|p{5cm}|p{5cm}|}
      \hline
      \multicolumn{3}{|c|}{\textbf{Nombre d'utilisateurs simultanés}} \\
      \hline
      \textbf{Test éffectué} & \textbf{Résultat attendu} & \textbf{Résultat observé} \\
      \hline

      On ce connecte à l'application avec 50 terminaux à la fois ou on simule ces 50 connexion&
      L'application reste stable et utilisable pour les 50 terminaux connectés&
      ... \\


      \hline

    \end{tabular}
  \end{center}

  \begin{center}
    \begin{tabular}{|p{5cm}|p{5cm}|p{5cm}|}
      \hline
      \multicolumn{3}{|c|}{\textbf{Volume de données}} \\
      \hline
      \textbf{Test éffectué} & \textbf{Résultat attendu} & \textbf{Résultat observé} \\
      \hline

<<<<<<< HEAD
      On crée 50 utilisateurs et on remplis pour chaqu utilisateurs le volume de données maximal (20Mo)&
=======
      On crée 50 utilisateurs et on rempli pour chaqu utilisateurs le volume de données maximal (20Mo)&
>>>>>>> 63e10de7ad9a5ea5d900f0df9458a8cdeca5ef6d
      Le système reste stable et chaques utilisateurs peut effectivement utiliser 20Mo de données&
      ... \\


      \hline

    \end{tabular}
  \end{center}

  \begin{center}
    \begin{tabular}{|p{5cm}|p{5cm}|p{5cm}|}
      \hline
      \multicolumn{3}{|c|}{\textbf{Accès au serveur}} \\
      \hline
      \textbf{Test éffectué} & \textbf{Résultat attendu} & \textbf{Résultat observé} \\
      \hline

      On fait 50 requètes en une seconde&
      Le système reste stable pour tout les utilisateurs connectés&
      ... \\


      \hline

    \end{tabular}
  \end{center}

  \begin{center}
    \begin{tabular}{|p{5cm}|p{5cm}|p{5cm}|}
      \hline
      \multicolumn{3}{|c|}{\textbf{Temps de latence}} \\
      \hline
      \textbf{Test éffectué} & \textbf{Résultat attendu} & \textbf{Résultat observé} \\
      \hline

<<<<<<< HEAD
      On fais 50 requète par secondes en continu et on mesure le temps entre la la requète et la réponse. On répète cette opération 1000 fois&
      On recoit la réponse en moins de 2 secondes dans au moins 900 cas&
=======
      On fais 50 requète par secondes en continu et on mesure le temps entre la requète et la réponse. On répète cette opération 1000 fois&
      On reçoit la réponse en moins de 2 secondes dans au moins 900 cas&
>>>>>>> 63e10de7ad9a5ea5d900f0df9458a8cdeca5ef6d
      ... \\

      \hline
      \hline
      On fais 150 requète par secondes en continu et on mesure le temps entre la requète et la réponse. On répète cette opération 1000 fois&
      On recoit la réponse en moins de 4 secondes dans au moins 750 cas&
      ... \\

      \hline

    \end{tabular}
  \end{center}

  \begin{center}
    \begin{tabular}{|p{5cm}|p{5cm}|p{5cm}|}
      \hline
<<<<<<< HEAD
      \multicolumn{3}{|c|}{\textbf{Temps d’execution des taches hors ligne}} \\
=======
      \multicolumn{3}{|c|}{\textbf{Temps d’execution des tâches hors ligne}} \\
>>>>>>> 63e10de7ad9a5ea5d900f0df9458a8cdeca5ef6d
      \hline
      \textbf{Test éffectué} & \textbf{Résultat attendu} & \textbf{Résultat observé} \\
      \hline

<<<<<<< HEAD
      Étant dans l'accueil, on va vers la page paramètres et on mesure le temps entre l'appuie sur le bouton et l'affichage de la page. On répète cette opération 1000 fois sur des smartphones différents&
      Le système affiche la page en 500 milisecondes dans au 650 cas&
=======
      Étant dans l'accueil, on va vers la page paramètres et on mesure le temps entre l'appui sur le bouton et l'affichage de la page. On répète cette opération 1000 fois sur des smartphones différents&
      Le système affiche la page en 500 milisecondes dans au moins 650 cas&
>>>>>>> 63e10de7ad9a5ea5d900f0df9458a8cdeca5ef6d
      ... \\


      \hline

    \end{tabular}
  \end{center}

  \begin{center}
    \begin{tabular}{|p{5cm}|p{5cm}|p{5cm}|}
      \hline
      \multicolumn{3}{|c|}{\textbf{Temps pour une prise de rendez-vous}} \\
      \hline
      \textbf{Test éffectué} & \textbf{Résultat attendu} & \textbf{Résultat observé} \\
      \hline

      On demande à 100 personnes de prendre un rendez-vous avec l'application (ces personnes connaissent déja l'application)&
      Au moins 60 personnes sur 100 prendrons le rendez-vous en moins de 5 minutes&
      ... \\


      \hline

    \end{tabular}
  \end{center}



  \end{document}
