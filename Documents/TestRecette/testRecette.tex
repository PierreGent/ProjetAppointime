\documentclass{article}
\usepackage[utf8]{inputenc}
\usepackage[T1]{fontenc}
\usepackage[frenchb]{babel}
\usepackage[bookmarks=true]{hyperref}
\usepackage{lmodern}
\usepackage{fullpage}

\author{Gentile Pierre, Didier-Roche François}
\date{\today}
\title{Cahier de test de recette}


\begin{document}

\maketitle

\begin{abstract}

  Ce document recense les tests unitaires qui doivent être menés lors de la phase de test de l'application Appointime. 
  Les grandes parties de l'application seront représentées dans ce document à l'aide de tableaux avec :
   le nom de la partie, les tests à effectuer, les résultats attendus et les résultats observés.

\end{abstract}

\newpage
\begin{center}
  \begin{tabular}{|p{4cm}|p{4cm}|p{4cm}|p{4cm}|}
    \hline
    \multicolumn{4}{|c|}{\textbf{Informations des utilisateurs (Inscription)}} \\
    \hline
    \textbf{Test éffectué} & \textbf{Résultat attendu} & \textbf{Résultat observé} & \textbf{Correction apportée}\\
    \hline

    J'entre des informations valides, l'email n'est pas déjà présent dans la base de données et j'appuie sur le bouton \og s'inscrire \fg{} &
    \begin{itemize}

      \item Le système me redirige vers la page de connexion en m'indiquant que mes informations ont bien été enregistrées.
      \item Les informations ont bien été inserées dans la base de données

      \end{itemize}&
      ...&
      ... \\

      \hline
      \hline
      J'entre des informations valides et l'email est déja présent dans la base de données puis j'appuie sur le bouton \og s'inscrire \fg{} &

      Le système me redirige vers la page d'inscription en m'indiquant que l'email est déjà utilisé &
      ...&
      ... \\


      \hline
      \hline
      J'entre une ou plusieurs informations invalides et j'appuie sur le bouton \og s'inscrire \fg{} &

      Le système me redirige vers la page d'inscription en m'indiquant les données invalides&
      ...&
      ... \\
      \hline

    \end{tabular}
  \end{center}


  \begin{center}
    \begin{tabular}{|p{4cm}|p{4cm}|p{4cm}|p{4cm}|}
      \hline

      \multicolumn{4}{|c|}{\textbf{Informations des utilisateurs (Connexion)}} \\
      \hline
      \textbf{Test éffectué} & \textbf{Résultat attendu} & \textbf{Résultat observé} & \textbf{Correction apportée} \\
      \hline

      J'entre des informations valides, la combinaison email+mot de passe est présente dans la base de données et j'appuie sur le bouton\og Se connecter \fg{}.&
      Le système me redirige vers la page d'accueil.&
      ...&
      ... \\

      \hline
      \hline
      J'entre des informations invalides et/ou la combinaison email+mot de passe n'est pas présente dans la base de données et j'appuie sur le bouton \og Se connecter \fg{}. &
      Le système me redirige vers la page de connexion en m'indiquant le(s) erreur(s).&
      ...&
      ... \\

      \hline
      \hline
      Pour les smartphones équipés, sur la page de connexion, je pose mon doigt sur le lecteur d'empreinte digitale alors que l'empreinte est enregistrée&

      Le système me redirige vers la page de d'accueil.&
      ...&
      ... \\

      \hline
      \hline
      Pour les smartphones équipés, sur la page de connexion, je pose mon doigt sur le lecteur d'empreinte digitale alors que l'empreinte n'est pas enregistrée&
	  Le système me redirige vers la page de connexion en m'indiquant une erreur.&

      ...&
      ... \\

      \hline
      \hline
      Pour les smartphones équipés, sur la page de connexion, j'utilise la reconnaissance faciale alors que mon visage est enregistré&

      Le système me redirige vers la page d'accueil&
      ...&
      ... \\

      \hline
      \hline
      Pour les smartphones équipés, sur la page de connexion, j'utilise la reconnaissance faciale alors que mon visage n'est pas enregistré&

      Le système me redirige vers la page de connexion en m'indiquant une erreur.&
      ...&
      ... \\


      \hline

    \end{tabular}
  \end{center}


  \begin{center}
    \begin{tabular}{|p{4cm}|p{4cm}|p{4cm}|p{4cm}|}
      \hline

      \multicolumn{4}{|c|}{\textbf{Informations des utilisateurs (Modification des informations)}} \\
      \hline
      \textbf{Test éffectué} & \textbf{Résultat attendu} & \textbf{Résultat observé} & \textbf{Correction apportée} \\
      \hline


      Je modifie une ou plusieurs informations en entrant des informations valides. Si je modifie l'email et que le nouvel email n'est pas déjà présent dans la base de données et j'appuie sur le bouton \og modifier \fg{}.&

      Le système me redirige vers la page de gestion de compte en m'indiquant que mes informations ont bien été modifiées. &
      ...&
      ... \\

      \hline
      \hline
      J'entre des informations valides mais le nouvel email est déjà présent dans la base de données et j'appuie sur le bouton \og modifier \fg{} &

      Le système me redirige vers la page de modification des informations en m'indiquant que l'email existe déjà &
      ...&
      ... \\

      \hline
      \hline
      J'entre des informations invalides.&

      Le système me redirige vers la page de modification des informations en m'indiquant les champs invalides.&
      ...&
      ... \\

      \hline

    \end{tabular}
  \end{center}


  \begin{center}
    \begin{tabular}{|p{4cm}|p{4cm}|p{4cm}|p{4cm}|}
      \hline
      \multicolumn{4}{|c|}{\textbf{Préférences des utilisateurs}} \\
      \hline
      \textbf{Test effectué} & \textbf{Résultat attendu} & \textbf{Résultat observé} & \textbf{Correction apportée} \\
      \hline

      En réglages basiques (notification activée), je recois une notification&
      Le smartphone m'affiche la notification dans le menu du smartphone et émet un son et/ou une vibration&
      ...&
      ... \\

      \hline
      \hline
      Je désactive les notifications en appuyant sur le bouton et je recois une notification&
      Le smartphone n'affiche aucune notification et n'émet aucun son&
      ...&
      ... \\


      \hline
      \hline
      Les notifications étant activées, je règle le son des notifications et je recois une notification&
      Le smartphone m'affiche la notification dans le menu du smartphone et émet un son qui sera plus ou moins fort selon la modification que j'ai éffectué&
      ...&
      ... \\

      \hline
      \hline
      Les notifications étant activées, je choisis une nouvelle tonalité de notifications et je recois une notification&
      Le smartphone m'affiche la notification dans le menu du smartphone et émet un son différent selon la modification que j'ai éffectué&

      ...&
      ... \\
 

      \hline
      \hline
      Etant dans le thème de base, j'active de thème nuit en appuyant sur le bouton prévu à cet effet.&
      Les couleurs de l'application changent quelque soit la page.&
      ...&
      ... \\



      \hline

    \end{tabular}
  \end{center}

  \begin{center}
    \begin{tabular}{|p{4cm}|p{4cm}|p{4cm}|p{4cm}|}
      \hline
      \multicolumn{4}{|c|}{\textbf{Recherche d'un professionnel}} \\
      \hline
      \textbf{Test effectué} & \textbf{Résultat attendu} & \textbf{Résultat observé} & \textbf{Correction apportée} \\

      \hline
      J'entre des mots clefs correspondant à une ou plusieurs entreprises&
      le système m'affiche une liste de liens vers les pages dédiées aux entreprises inscrites dans la base de données et correspondant à la recherche&
      ...&
      ... \\

      \hline
      \hline
      J'utilise mon appareil photo pour analyser un flash code correspondant à une entreprise&
     Le système affiche le lien de la page dédiée à l'entreprise correspondant au flashcode&
      ...&
      ... \\

      \hline
      \hline
      J'entre des mots clefs ne correspondant à aucune entreprise&
      Le système m'indique qu'aucun résultat n'a été trouvé&
      ...&
      ... \\

      \hline
      \hline
      J'utilise mon appareil photo pour annalyser un flash code qui n'est pas présent dans la base de données&
      Le système m'indique qu'aucun résultat n'à été trouvé &
      ...&
      ... \\

     

      



      \hline

    \end{tabular}
  \end{center}


  \begin{center}
    \begin{tabular}{|p{4cm}|p{4cm}|p{4cm}|p{4cm}|}
      \hline

      \multicolumn{4}{|c|}{\textbf{Prise de rendez-vous chez un professionnel}} \\
      \hline
      \textbf{Test effectué} & \textbf{Résultat attendu} & \textbf{Résultat observé}  & \textbf{Correction apportée}\\
      \hline

      Je séléctionne une prestation dans le menu déroulant de la page prévue à cet effet&
      Le système m'affiche le calendrier du professionnel avec les créneaux disponibles supérieurs ou égaux à la durée de la prestation.&
      ...&
      ... \\

      \hline
      \hline

      Une fois le calendrier affiché, j'appuie sur un créneau disponible&
      Le créneau change de couleur afin d'indiquer qu'il est sélectionné&
      ...&
      ... \\
	\hline
      \hline


      Tant qu'aucun créneau n'est séléctionné&
      le bouton valider n'est pas utilisable&
      ...&
      ... \\
      
       \hline
      \hline
      Une fois le créneau sélectionné, j'appuie sur le bouton réserver&
      Le système me redirige vers la page de confirmation&
      ...&
      ... \\
     
      \hline
      \hline
      Etant sur la page de confirmation,j'annule ma réservation&
      Le système me redirige vers la page précédente&
      ...&
      ... \\

		\hline
      \hline
      Etant un particulier ayant une réservation active, je programme un rappel&
      Le rappel a bien lieu au jour/à l'heure prévu(e)
      ...&
      ... \\

      \hline

    \end{tabular}
  \end{center}

  \begin{center}
    \begin{tabular}{|p{4cm}|p{4cm}|p{4cm}|p{4cm}|}
      \hline
      \multicolumn{4}{|c|}{\textbf{Parametrage des prestations d'un professionnel}} \\
      \hline
      \textbf{Test effectué} & \textbf{Résultat attendu} & \textbf{Résultat observé}  & \textbf{Correction apportée}\\
      \hline

      Dans le menu de paramétrage des prestations j'appuie sur le bouton \og ajouter une prestation \fg{}&
      Le système me redirige vers le formulaire d'ajout de prestation&
      ...&
      ... \\

      \hline
      \hline

      Dans le formulaire d'ajout de prestation, j'entre des informations
      valides et j'appuie sur le bouton \og Valider \fg{}&
      Le système ajoute les informations en base de données et me redirige
      vers le menu
      de paramétrage

      des prestations&
      ...&
      ... \\

      \hline
      \hline

      Dans le formulaire d'ajout de prestation j'entre des informations
      invalides et j'appuie sur \og Valider \fg{} &
      Le système m'indique les erreurs que j'ai fait et me renvoie sur le
      formulaire d'ajout de prestation &
      ...&
      ... \\

      \hline
      \hline
      Dans le menu de parametrage des prestations j'appuie sur le bouton \og
      Modifier \fg{} de n'importe quelle prestation&
      Le système me redirige vers le formulaire de modification de prestation&
      ...&
      ... \\

      \hline
      \hline
      Dans le formulaire de modification de prestation, j'entre des
      informations valides et j'appuie sur le boutton \og Modifier \fg{}&
      Le système modifie les informations en base de données et me redirige
      vers la section de gestion des prestations.&
      ...&
      ... \\

      \hline
      \hline
      Dans le formulaire d'ajout de prestation, j'entre des informations

      invalides et j'appuie sur le bouton \og Valider \fg{}&
      Le système m'indique les erreurs que j'ai fait et me renvoie sur le
      formulaire de modification de prestation&
      ...&
      ... \\

      \hline
      \hline
      Dans la section de gestion des prestations j'appuie sur le bouton \og
      Supprimer \fg{} de n'importe quelle prestation&
      Le système supprime la prestation en base de données et je ne la vois plus
      apparaître dans
      la liste. &
      ...&
      ... \\



      \hline

    \end{tabular}
  \end{center}





\begin{center}
    \begin{tabular}{|p{4cm}|p{4cm}|p{4cm}|p{4cm}|}
      \hline
      \multicolumn{4}{|c|}{\textbf{Paramétrage de l'entreprise d'un professionnel}} \\
      \hline
      \textbf{Test effectué} & \textbf{Résultat attendu} & \textbf{Résultat observé} & \textbf{Correction apportée} \\
      \hline

      Dans le menu de paramétrage d'une entreprise j'appuie sur le bouton \og modifier les informations \fg{}&
      Le système me redirige vers le formulaire de modification des informations de l'entreprise&
      ...&
      ... \\

      \hline
      \hline

      Dans le formulaire de gestion des informations sur l'entreprise, j'entre des informations
      valides et j'appuie sur le boutton \og Valider \fg{}&
      Le système ajoute les informations en base de données et me redirige
      vers le menu d'affichage des informations sur l'entreprise&
      ...&
      ... \\

      \hline
      \hline

      Dans le formulaire de gestion des informations sur l'entreprise, j'entre des informations
      invalides et j'appuie sur \og Valider \fg{} &
      Le système m'indique les erreurs que j'ai fait et me renvoie sur le
      formulaire  de gestion des informations sur l'entreprise&
      ...&
      ... \\

   


      \hline

    \end{tabular}
  \end{center}







\begin{center}
   \begin{tabular}{|p{4cm}|p{4cm}|p{4cm}|p{4cm}|}
      \hline
      \multicolumn{4}{|c|}{\textbf{Paramétrage du calendrier d'un professionnel}} \\
      \hline
      \textbf{Test effectué} & \textbf{Résultat attendu} & \textbf{Résultat observé} & \textbf{Correction apportée} \\
      \hline

      Dans le menu de paramétrage du calendrier j'appuie sur le bouton \og paramétrer les horaires d'ouverture \fg{}&
      Le système me redirige vers le formulaire de gestion des horaires d'ouverture&
      ...&
      ... \\

      \hline
      \hline

      Dans le formulaire de gestion des horaires d'ouverture, j'entre des informations
      valides et j'appuie sur le boutton \og Valider \fg{}&
      Le système ajoute les informations en base de données et me redirige
      vers le menu de gestion des horaires d'ouverture&
      ...&
      ... \\

      \hline
      \hline

      Dans le formulaire de gestion des horaires d'ouverture, j'entre des informations
      invalides et j'appuie sur \og Valider \fg{} &
      Le système m'indique les erreurs que j'ai fait et me renvoie sur le
      formulaire de gestion des horaires d'ouverture,&
      ...&
      ... \\

      \hline
      \hline
      Dans le menu de paramétrage du calendrier j'appuie sur le bouton \og
      Modifier \fg{} de n'importe quel jour&
      Le système me redirige vers le formulaire de modification des horaires pour ce jour&
      ...&
      ... \\

      \hline
      \hline
      Dans le formulaire de modification d'horaires d'ouverture du jour, j'entre des
      informations valides et j'appuie sur le boutton \og Modifier \fg{}&
      Le système modifie les informations en base de données et me redirige
      vers
      le
      menu de gestion des horaires d'ouverture.&
      ...&
      ... \\

      \hline
      \hline
      Dans le formulaire de gestion des horaires d'ouverture du jour, j'entre des informations
      invalides et j'appuie sur le boutton \og Valider \fg{}&
      Le système m'indique les erreurs que j'ai fait et me renvoie sur le
      formulaire de modifications d'horaires du jour &
      ...&
      ... \\

    


      \hline

    \end{tabular}
  \end{center}














  \begin{center}
    \begin{tabular}{|p{4cm}|p{4cm}|p{4cm}|p{4cm}|}
      \hline
      \multicolumn{4}{|c|}{\textbf{Nombre d'utilisateurs simultanés}} \\
      \hline
      \textbf{Test éffectué} & \textbf{Résultat attendu} & \textbf{Résultat observé} & \textbf{Correction apportée} \\
      \hline

      On se connecte à l'application avec 50 terminaux à la fois ou on simule ces 50 connexion&
      L'application reste stable et utilisable pour les 50 terminaux connectés&
      ...&
      ... \\


      \hline

    \end{tabular}
  \end{center}

  \begin{center}
    \begin{tabular}{|p{4cm}|p{4cm}|p{4cm}|p{4cm}|}
      \hline
      \multicolumn{4}{|c|}{\textbf{Volume de données}} \\
      \hline
      \textbf{Test effectué} & \textbf{Résultat attendu} & \textbf{Résultat observé} & \textbf{Correction apportée} \\
      \hline

      On crée 50 utilisateurs et on atteind pour chaque utilisateur le volume de données maximal (20Mo)&
      Le système reste stable et chaque utilisateur peut effectivement utiliser 20Mo de données&
      ...&
      ... \\


      \hline

    \end{tabular}
  \end{center}

  \begin{center}
    \begin{tabular}{|p{4cm}|p{4cm}|p{4cm}|p{4cm}|}
      \hline
      \multicolumn{4}{|c|}{\textbf{Accès au serveur}} \\
      \hline
      \textbf{Test effectué} & \textbf{Résultat attendu} & \textbf{Résultat observé} & \textbf{Correction apportée} \\
      \hline

      On fait 50 requêtes en une seconde&
      Le système reste stable pour tous les utilisateurs connectés&
      ...&
      ... \\


      \hline

    \end{tabular}
  \end{center}

  \begin{center}
    \begin{tabular}{|p{4cm}|p{4cm}|p{4cm}|p{4cm}|}
      \hline
      \multicolumn{4}{|c|}{\textbf{Temps de latence}} \\
      \hline
      \textbf{Test effectué} & \textbf{Résultat attendu} & \textbf{Résultat observé} & \textbf{Correction apportée} \\
      \hline

      On fait 50 requêtes par seconde en continu et on mesure le temps entre la requête et la réponse. On répète cette opération 1000 fois&
      On reçoit la réponse en moins de 2 secondes dans au moins 900 cas&
      ...&
      ... \\

      \hline
      \hline
      On fait 150 requêtes par seconde en continu et on mesure le temps entre la requête et la réponse. On répète cette opération 1000 fois&
      On recoit la réponse en moins de 4 secondes dans au moins 750 cas&
      ...&
      ... \\

      \hline

    \end{tabular}
  \end{center}

  \begin{center}
    \begin{tabular}{|p{4cm}|p{4cm}|p{4cm}|p{4cm}|}
      \hline
      \multicolumn{4}{|c|}{\textbf{Temps d’execution des tâches hors ligne}} \\
      \hline
      \textbf{Test effectué} & \textbf{Résultat attendu} & \textbf{Résultat observé} & \textbf{Correction apportée} \\
      \hline


      Étant sur la page d'accueil, on va vers la page paramètres et on mesure le temps entre l'appui sur le bouton et l'affichage de la page. On répète cette opération 1000 fois sur des smartphones différents&
      Le système affiche la page en 500 millisecondes dans au moins 650 cas&
      ...&
      ... \\


      \hline

    \end{tabular}
  \end{center}

  \begin{center}
    \begin{tabular}{|p{4cm}|p{4cm}|p{4cm}|p{4cm}|}
      \hline
      \multicolumn{4}{|c|}{\textbf{Temps pour une prise de rendez-vous}} \\
      \hline
      \textbf{Test effectué} & \textbf{Résultat attendu} & \textbf{Résultat observé}  & \textbf{Correction apportée}\\
      \hline

      On demande à 100 personnes de prendre un rendez-vous avec l'application (ces personnes connaissent déjà l'application)&
      Au moins 60 personnes sur 100 prendront le rendez-vous en moins de 5 minutes&
      ...&
      ... \\


      \hline

    \end{tabular}
  \end{center}



  \end{document}
